%	SECTION TITLE
%-------------------------------------------------------------------------------
\cvsection{Projects}
% Maze solver. Bluetooth car. AUVSI? Kiri!!

%-------------------------------------------------------------------------------
%	CONTENT
%-------------------------------------------------------------------------------
\cvsubsection{AUVSI Student Unmanned Aerial Systems Competition}
\begin{cvparagraph}
Developed a robust RRT path planner for the AUVSI SUAS competition. This planner avoided obstacles while minimizing the waypoint capture error through ensuring long straight paths through waypoints. Also fabricated and repaired many fixed-wing UAVs, created an image distortion correction program for letter and shape recognition, and many other tasks over the three years on the team. 
\end{cvparagraph}

%\cvsubsection{Flight dynamics class project or just AUVSI}
%\begin{cvparagraph}
%Developed an Unscented Kalman Filter for fixed-wing attitude estimation. Implemented this estimator on a Robot Operating System simulation of a micro unmanned air vehicle with noisy accelerometers and gyros.
%\end{cvparagraph}

\cvsubsection{Autopilot Implementation}
\begin{cvparagraph}
 Implemented the autopilot from \textit{Small Unmanned Aircraft: Theory and Practice} in Python. This includes an implementation of the controller, estimator, path planner, and path manager.
\end{cvparagraph}

%\cvsubsection{Twitter Bot}
%\begin{cvparagraph}
%Sentiment analysis. Designed an operational space controller for inverted pendulum control on an arm of Rethink Robotics' Baxter unit. Implemented this controller in the Robot Operating System and Gazebo using Rethink Robotics' Baxter SDK.
%\end{cvparagraph}


%
%\cvsubsection{Deep Q Network}
%\begin{cvparagraph}
%  Replicated Google Deepmind's work on the Deep Q Network by training a reinforcement learning algorithm to play space invaders using only visual data.
%\end{cvparagraph}
%
%\cvsubsection{Heartbeat Segmentation \& Anomaly Detection}
%\begin{cvparagraph}
%  Segmented phonocardiograms and electrocardiograms into four states: S1, Systole, S2 and Diastole. This was done using a hidden semi-markov model and the associated Viterbi algorithm. Used these segmented heartbeats in anomaly detection using various classifiers: support vector machines, gaussian discriminant analysis, and multi-layer perceptron classifiers.
%\end{cvparagraph}
